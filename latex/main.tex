\documentclass[12pt]{article}
\usepackage{geometry}
\usepackage{fontspec}
\usepackage{hyperref}
\setmainfont{Noto Sans CJK SC}

\title{日本麻将完全攻略}
\date{} 

\begin{document}

\maketitle
\tableofcontents

% BEGIN README CONTENT (Convert markdown headings to LaTeX; body omitted for brevity)
\section{1. 大局观}
  \subsection{1.1 什么是麻将}
    \subsubsection{1.1.1 面子:顺子、刻子——麻将大厦的承重墙}
    % ...
  \subsection{1.2 日本麻将与其他麻将的区别}
    % ...
  \subsection{1.3 段位系统——衡量雀力的标尺}
    % ...
\section{2. 如何最快做出满贯——速度与打点的平衡艺术}
  % ...
\section{3. 如何最快听牌(不以和牌为目标):牌效}
  % ...
\section{4. 被动安全打法}
  % ...
\section{5. 犬之道:平衡的艺术}
  % ...
\section*{结语:麻将的道与术}
% ...
\bibliographystyle{plain}
\bibliography{refs}

\end{document}